\zihao{-4}

\hspace{-0.85cm}院\ \ 系\underline{\makebox[6.7cm]{物理科学与技术学院}}
专\ \ 业\underline{\makebox[6.7cm]{应用物理学}}
\\年\ \ 级\underline{\makebox[6.7cm]{2018级}}
姓\ \ 名\underline{\makebox[6.7cm]{萧贺源}}
\\题\ \ 目\underline{\makebox[14.65cm]{碳-氮化硼纳米管超晶格的传热与力学性质研究}}
\\
\\
指导教师
\\评\quad\quad 语\uline{\hbox to 13.95cm{}}

\hspace{0.85cm}\underline{\hbox to 13.95cm{}}

\hspace{0.85cm}\underline{\hbox to 13.95cm{}}

\hspace{0.85cm}\underline{\hbox to 13.95cm{}}

\hspace{0.85cm}\underline{\hbox to 13.95cm{}}

\hspace{0.85cm}\underline{\hbox to 13.95cm{}}

\hspace{0.85cm}\underline{\hbox to 13.95cm{}}

\hspace{0.85cm}\underline{\hbox to 13.95cm{}}

\hspace{0.85cm}\underline{\hbox to 13.95cm{}}
\\

\hspace{9.8cm}指导教师\underline{\hbox to 2cm{}}(签章)
\\
\\
评\ \ 阅\ \ 人
\\评\quad\quad 语\underline{\hbox to 13.95cm{}}

\hspace{0.85cm}\underline{\hbox to 13.95cm{}}

\hspace{0.85cm}\underline{\hbox to 13.95cm{}}

\hspace{0.85cm}\underline{\hbox to 13.95cm{}}

\hspace{0.85cm}\underline{\hbox to 13.95cm{}}

\hspace{0.85cm}\underline{\hbox to 13.95cm{}}

\hspace{0.85cm}\underline{\hbox to 13.95cm{}}

\hspace{0.85cm}\underline{\hbox to 13.95cm{}}

\hspace{0.85cm}\underline{\hbox to 13.95cm{}}
\\

\hspace{9.8cm}评\ \ 阅\ \ 人\underline{\hbox to 2cm{}}(签章)
\\

\hspace{-0.85cm}成绩\underline{\hbox to 5.5cm{}}
\\
\\答辩委员会主任\underline{\hbox to 2cm{}}(签章)
\\

\hfill \hspace{2cm}年\hspace{1cm}月\hspace{1cm}日

\newpage
\vspace*{0.5cm}
\begin{center}{\hei \zihao{-2} \textbf{毕业设计(论文)任务书}}\end{center}
\vspace{1cm}

\hspace{-0.85cm}班\ \ 级\underline{\makebox[3.91cm]{物理2018-01班}}
学生姓名\underline{\makebox[3.91cm]{萧贺源}}
学\ \ 号\underline{\makebox[3.91cm]{2018115237}}
\\发题日期:\hspace{2cm}年\hspace{1cm}月\hspace{1cm}日\hfill 完成日期:\hspace{2cm}年\hspace{1cm}月\hspace{1cm}日
\\
\\题\ \ 目\underline{\makebox[14.55cm]{碳-氮化硼纳米管超晶格的传热与力学性质研究}}
\\1、本设计(论文)的目的、意义

\uline{本论文拟研究碳-氮化硼纳米管超晶格的热传导性质。纳米材料随着空间维度的降低,其热传导性质也发生着明显的变化。超晶格材料由于存在多个界面,其热传导性质与单一组分材料相比有明显的差异。本论文拟使用分子动力学模拟,对碳-氮化硼纳米管超晶格的热传导性质进行研究,对纳米级电子材料的设计提供有用的信息。}
\\2、学生应完成的任务

\uline{(1)广泛调研及阅读文献,熟悉利用分子动力学计算材料热导率的方法;\\                    
(2)设计纳米管超晶格材料的结构;\\      
(3)计算相关结构的热导率;\\      
(4) 分析纳米结构对相关材料热导率影响的规律。}
\\3、本设计(论文)与本专业的毕业要求达成度如何?(如在知识结构、能力结构、素质结构等方面有哪些有效的训练。)

\uline{本课题是在所学的热力学与统计物理、固体物理等知识上的进一步巩固、拓展和深化,有效提高学生应用相关物理知识的能力,同时对学生独立开展工作的能力有较好地培养,初步从事科学研究,进一步开拓学生科学视野,提高分析问题、解决问题的能力,促进对科学研究的兴趣。}

\newpage

\hspace{-0.85cm}4、本设计(论文)各部分内容及时间分配:(共\underline{\makebox[1cm]{17}}周)
\\第一部分\uline{\quad 调研分子动力学计算材料热导率的方法,完成文献翻译}\hfill(1-3周) 
\\第二部分\uline{\quad 建立不同纳米管超晶格的模型}\hfill(4-8周) 
\\第三部分\uline{\quad 计算相关模型的热导率和力学性质}\hfill(9-11周)
\\第四部分\uline{\quad 分析周期长度对超晶格热导率影响的规律。}\hfill(12-14周)
\\第五部分\uline{\quad 论文撰写}\hfill(15周)
\\评阅及答辩\hfill(1周)
\vspace{16cm}
\\指导教师:\hspace{4cm}年\hspace{1cm}月\hspace{1cm}日
\\审\ \ 批\ \ 人:\hspace{4cm}年\hspace{1cm}月\hspace{1cm}日